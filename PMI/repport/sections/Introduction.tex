\section*{Introduction}
\addcontentsline{toc}{section}{\protect\numberline{}Introduction}
\label{sec:introduction}

Ce document présente le rapport final du projet de \acrfull{PMI} réalisé dans le cadre de la troisième année à l'\acrshort{IPSA}
dans l'année académique 2025-2026. Le projet, intitulé \textit{ET2 - Perspective d’expérience d’un système de séparation
d’étages de fusée par aérofreinage}, vise à concevoir et analyser un système innovant de séparation d’étages pour une fusée
bi-étage, en utilisant la technique de l’aérofreinage. Ce projet représente une partie d'un projet plus vaste mené par
l'associationn AéroIPSA : le projet SP-02.

\subsection*{1 - AéroIPSA}

Créée il y a plus de trente ans, AéroIPSA est l’association aérospatiale de l’école d’ingénieurs IPSA. Elle a pour mission de
concevoir, développer et mettre en œuvre des projets techniques complets dans le domaine de l’aéronautique et du spatial. Les
activités de l’association couvrent l’ensemble du cycle de réalisation — de la conception mécanique et électronique à la
fabrication et aux essais — autour de projets tels que des fusées expérimentales, des Cansats ou des systèmes embarqués. Ces
projets constituent un cadre d’application concret des enseignements de l’école et offrent aux étudiants une première expérience
de gestion technique, humaine et opérationnelle.

AéroIPSA s’attache également à favoriser la transmission des connaissances entre promotions et la montée en compétence de ses
membres à travers un encadrement technique, des formations internes et des travaux collaboratifs. Chaque année, ces efforts se
concrétisent par la participation de l’association à la campagne nationale de lancements du \gls{C'Space}, organisée par le
\acrfull{CNES} et l'association Planète Sciences, où les étudiants valident et présentent les projets réalisés.

\subsection*{2 - Projet SP-02}

SP-02 constitue l'un des nombreux projet de fusée développé par l’association AéroIPSA. Ce projet ambitieux vise à concevoir et
réaliser une fusée expérimentale bi-étage, destinée à participer à la campagne C'Space 2026. Il a pour finalité la conception,
la réalisation et la qualification d’un lanceur à deux étages destiné à valider plusieurs technologies innovantes appliquées
aux systèmes de séparation et d’avionique embarquée.

L’objectif principal est de démontrer la faisabilité d’une séparation inter-étage par freinage aérodynamique, alternative aux
méthodes pyrotechniques ou mécaniques conventionnelles. Cette architecture a pour but de simplifier la chaîne d’activation tout
en réduisant les risques liés aux opérations pyrotechniques. Parallèlement, la mission embarque un ensemble de mesures
avioniques dédiées à la caractérisation du vol — accélérations, attitude, pression et altitude — dont les données sont
enregistrées et transmises en temps réel par télémesure vers la station sol.

Le développement du SP-02 s’appuie sur une démarche d’ingénierie système complète, intégrant les volets mécanique, électronique,
logiciel et essais, conformément aux spécifications du cahier des charges \acrshort{CNES}-Planète Sciences pour les fusées
bi-étages. Ce projet vise ainsi à approfondir la maîtrise des processus de conception et de validation expérimentale propres à
l’aéronautique et au spatial, tout en contribuant à l’enrichissement du savoir-faire technique au sein de l’association AéroIPSA.

\subsection*{3 - Rôle du PMI}

Le projet de \acrshort{PMI} s'inscrit dans le cadre plus large du projet SP-02 d'AéroIPSA, en se concentrant sur l'étude et la
conception du système de séparation d'étages par aérofreinage. Le rôle principal du \acrshort{PMI} est de développer une
étude détaillées de ce système innovant, en évaluant ses performances, sa fiabilité et son intégration au sein de la fusée
courante SP-02, tout en laissant la possibilité d'une application future sur d'autres projets de lanceurs, à des échelles et des
domaines de vol différents.

\subsection*{4 - Contexte de réalisation}

Le projet de \acrshort{PMI} a été réalisé au cours de l'année académique 2025-2026, en parallèle des autres activités scolaires
et associatives des membres d'AéroIPSA. La gestion du temps et des ressources a été un aspect crucial, nécessitant une
organisation rigoureuse pour concilier les exigences académiques et les objectifs du projet. Le travail a été mené en étroite
collaboration avec les autres membres de l'équipe SP-02, ainsi qu'avec des experts techniques et des encadrants de l'école, afin
d'assurer la cohérence et la qualité des livrables.

La réalisation du \acrshort{PMI} a dû prendre en compte une contrainte temporelle stricte : le calendrier du \acrshort{PMI}
s'achève fin janvier 2026, alors que le projet SP-02 dispose d'un horizon opérationnel étendu jusqu'à fin juin 2026. Cette
différence d'échéances impose une planification accélérée, des jalons serrés et une coordination renforcée afin de fournir des
livrables exploitables par l'équipe SP-02 pour la suite du développement.

\newpage
