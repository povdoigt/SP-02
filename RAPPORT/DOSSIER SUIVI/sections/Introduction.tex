\section*{Introduction}
\addcontentsline{toc}{section}{\protect\numberline{}Introduction}
\label{sec:introduction}

Ce document constitue le dossier de suivi du projet SP-02, développé par l’association AéroIPSA au sein de l’école d’ingénieurs
\arcshort{IPSA}. 

\subsection*{1 - AéroIPSA}

Créée il y a plus de trente ans, AéroIPSA est l’association aérospatiale de l’école d’ingénieurs IPSA. Elle a pour mission de
concevoir, développer et mettre en œuvre des projets techniques complets dans le domaine de l’aéronautique et du spatial. Les
activités de l’association couvrent l’ensemble du cycle de réalisation — de la conception mécanique et électronique à la
fabrication et aux essais — autour de projets tels que des fusées expérimentales, des Cansats ou des systèmes embarqués. Ces
projets constituent un cadre d’application concret des enseignements de l’école et offrent aux étudiants une première expérience
de gestion technique, humaine et opérationnelle.

AéroIPSA s’attache également à favoriser la transmission des connaissances entre promotions et la montée en compétence de ses
membres à travers un encadrement technique, des formations internes et des travaux collaboratifs. Chaque année, ces efforts se
concrétisent par la participation de l’association à la campagne nationale de lancements du \gls{C'Space}, organisée par le
\acrfull{CNES} et l'association Planète Sciences, où les étudiants valident et présentent les projets réalisés.

\subsection*{2 - Projet SP-02}

SP-02 constitue l'un des nombreux projet de fusée développé par l’association AéroIPSA. Ce projet ambitieux vise à concevoir et
réaliser une fusée expérimentale bi-étage, destinée à participer à la campagne C'Space 2026. Il a pour finalité la conception,
la réalisation et la qualification d’un lanceur à deux étages destiné à valider plusieurs technologies innovantes appliquées
aux systèmes de séparation et d’avionique embarquée.

L’objectif principal est de démontrer la faisabilité d’une séparation inter-étage par freinage aérodynamique, alternative aux
méthodes pyrotechniques ou mécaniques conventionnelles. Cette architecture a pour but de simplifier la chaîne d’activation tout
en réduisant les risques liés aux opérations pyrotechniques. Parallèlement, la mission embarque un ensemble de mesures
avioniques dédiées à la caractérisation du vol — accélérations, attitude, pression et altitude — dont les données sont
enregistrées et transmises en temps réel par télémesure vers la station sol.

Le développement du SP-02 s’appuie sur une démarche d’ingénierie système complète, intégrant les volets mécanique, électronique,
logiciel et essais, conformément aux spécifications du cahier des charges \acrshort{CNES}-Planète Sciences pour les fusées
bi-étages. Ce projet vise ainsi à approfondir la maîtrise des processus de conception et de validation expérimentale propres à
l’aéronautique et au spatial, tout en contribuant à l’enrichissement du savoir-faire technique au sein de l’association AéroIPSA.

\newpage
